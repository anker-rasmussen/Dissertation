\documentclass[12pt,a4paper]{article}
\usepackage[margin=2.5cm]{geometry}
\usepackage{setspace}
\usepackage{titlesec}
\usepackage{enumitem}
\usepackage{hyperref}
\usepackage{graphicx}
\usepackage{titling}

\setstretch{1.25}
\setlength{\parskip}{0.8em}
\setlength{\parindent}{0pt}
\hyphenpenalty=10000
\exhyphenpenalty=10000

\titleformat{\section}{\large\bfseries}{\thesection}{1em}{}
\titleformat{\subsection}{\normalsize\bfseries}{\thesubsection}{1em}{}
\titleformat{\subsubsection}{\normalsize\itshape}{\thesubsubsection}{1em}{}

\pretitle{
    \begin{center}
        \vspace*{-3cm}
    \includegraphics[width=0.35\textwidth]{UniLogo.png}\\[1.5cm]
    \LARGE\bfseries
}
\posttitle{\par\end{center}\vskip 1cm}
\title{Project Definition Document}
\author{}
\date{\today}
\begin{document}

\maketitle
\vspace{0.5cm}

\begin{center}
\textbf{Project Title:} {Local-First Collaborative Text Editor using the Veilid Framework} \\[1cm]
\textbf{Degree Programme:} {BSc (Hons) Computer Science} \\[1cm]
\textbf{Project Consultant:} {Martin Nyx Brain} \\[1cm]
\textbf{By:}{Anker Rasmussen} \\[0.2cm]
{\small \textbf{anker.rasmussen@city.ac.uk}} \\ [0.3cm]
\textbf{Category:} Academic Client Project \\[0.3cm]
{\small \textbf{Subcategory:} Application Development} \\[1cm]
\textbf {Word count:}
\end{center}

\newpage


\section{Proposal}

\subsection{Problem to be Solved}
Current collaborative editors like Google Docs and Microsoft 365 rely on centralized 
infrastructure to handle collaborative editing of documents. This centralized approach to
storage and coordination introduces risks in three main fields: user privacy, data
ownership, and availability. This project will explore how to enable secure, consistent,
and private document sharing between peers can be accomplished using Veilid's peer-to-peer
architecture and CRDT-based state synchronisation.


\subsection{Project Objectives}
\begin{itemize}[noitemsep]
    \item Create 
    \item Objective 2
    \item Objective 3
\end{itemize}

\subsection{Project Beneficiaries}
Identify who benefits from this project (users, researchers, organizations, etc.) and how they benefit.

\subsection{Project Plan}
Provide a high-level timeline of the project. You can use a list or table if preferred.

\begin{itemize}[noitemsep]
    \item Placeholder0
    \item Placeholder1
    \item Placeholder2
    \item Placeholder3
\end{itemize}

\subsection{Risks Affecting the Project}
Outline any technical, logistical, or ethical risks that may affect project success, and how you plan to mitigate them.

\subsection{Legal, Social, Ethical and Professional Considerations}
Discuss any relevant ethical or legal implications of your project (e.g., data privacy, bias, intellectual property, accessibility).

\subsection{References}
\begin{itemize}[noitemsep]
    \item Author, A. (Year). \textit{Title of the paper}. Journal/Source.
    \item Author, B. (Year). \textit{Title of another reference}.
\end{itemize}

\newpage
\section{Research Ethics Checklist}
Summarize any ethical considerations and indicate whether formal ethics approval is needed. If applicable, reference consent forms or procedures.

\newpage
\section{Client Information Sheet (External Client Projects Only)}
Provide details about the client organization, contact person, and nature of collaboration.

\newpage
\section{Appendix: Use of Generative AI (if applicable)}
If AI tools (e.g., ChatGPT, Copilot) were used in preparing this document or project materials, describe exactly how and to what extent they were used.

\newpage
\section{Appendix: Legal, Social, Ethical, and Professional Issues (LSEPI)}
For Category 3 projects, provide a more detailed analysis of relevant LSEPI topics.

\end{document}
