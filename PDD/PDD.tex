\documentclass[12pt,a4paper]{article}
\usepackage[margin=2.5cm]{geometry}
\usepackage{setspace}
\usepackage{titlesec}
\usepackage{enumitem}
\usepackage[hidelinks]{hyperref}
\usepackage{graphicx}
\usepackage{titling}
\usepackage[
  backend=biber,
  style=ieee,
  sorting=none,
  maxbibnames=99
]{biblatex}
\immediate\write18{texcount -1 -sum \jobname.tex > \jobname.wc}
\newcommand{\wordcount}{\input{\jobname.wc}}

\addbibresource{dissertation-sources.bib}
\DefineBibliographyExtras{english}{\let\finalandcomma=\empty}


\setstretch{1.25}
\setlength{\parskip}{0.8em}
\setlength{\parindent}{0pt}
\hyphenpenalty=10000
\exhyphenpenalty=10000

\titleformat{\section}{\large\bfseries}{\thesection}{1em}{}
\titleformat{\subsection}{\normalsize\bfseries}{\thesubsection}{1em}{}
\titleformat{\subsubsection}{\normalsize\itshape}{\thesubsubsection}{1em}{}

\pretitle{
    \begin{center}
        \vspace*{-3cm}
    \includegraphics[width=0.35\textwidth]{UniLogo.png}\\[1.5cm]
    \LARGE\bfseries
}
\posttitle{\par\end{center}\vskip 1cm}
\title{Project Definition Document}
\author{}
\date{\today}
\begin{document}

\maketitle

\begin{center}
\textbf {Project Title:} {Decentralized Peer-to-Peer SMPC Sealed-bid auction app over the Veilid framework} \\[1cm]
\textbf{Degree Programme:} {BSc (Hons) Computer Science} \\[1cm]
\textbf{Project Consultant:} {Martin Nyx Brain} \\[1cm]
\textbf{By:}{Anker Rasmussen} \\[0.2cm]
{\small \textbf{anker.rasmussen@city.ac.uk}} \\ [0.3cm]
\textbf{Category:} Academic Client Project \\[0.3cm]
{\small \textbf{Subcategory:} Application Development} \\[1cm]
\textbf {Word count:}\wordcount\\[0.3cm]
\textbf {Proprietary Interests:} None
\end{center}

\newpage


\section{Proposal}

\subsection{Problem to be Solved}
Popular online consumer marketplaces either use \emph{open, incrementally visible bidding} or \emph{no auctions at all}, which undermines sealed-bid privacy and fairness. eBay’s proxy-bidding model reveals live bid progression and awards the item to the highest bid at close \parencite{ebay_help_bidding}. Facebook Marketplace, by contrast, is a listing-plus-messaging workflow rather than a formal auction protocol, with weak buyer protections \parencite{fb_marketplace_help}. Public blockchains add strong auditability but invite transaction ordering attacks (front-running/MEV) that can leak or distort bids during submission and reveal phases \parencite{daian2019_flashboys2}. Cryptographic sealed-bid auctions can hide non-winning bids entirely; deploying them without a trusted auctioneer requires secure multiparty computation (MPC). MPC has been demonstrated in production (e.g., the Danish sugar-beet auction) and modern frameworks such as MP-SPDZ show practical performance across secret-sharing, HE, and garbled-circuit backends \parencite{bogetoft2009_mpc_live,mp-spdz,damgaard2012_spdz,evans2018_pragmatic_mpc,sako2000_hide_losers}.

\subsection*{What is Veilid?}
\emph{Veilid} is a privacy-first, open-source, peer-to-peer application framework. Each app embeds a node into a global overlay where peers are equal (no privileged relays), connections are end-to-end encrypted, and private routing obscures network locations. After a brief bootstrap, apps communicate directly over transports such as UDP/TCP/QUIC/Web and exchange data via a secure DHT designed for mobile and desktop \parencite{veilid_developer_book}. In short: an application overlay providing addressable, encrypted endpoints (public-key identities) and metadata-minimizing communication, independent of any blockchain.

\subsection*{Why Veilid for this marketplace?}
The marketplace requires private identities, censorship-resistant transport, and NAT-friendly reachability without a trusted coordinator. Veilid offers: (i) addressable, encrypted endpoints keyed by public keys (not IPs), (ii) a secure DHT for publishing listings and locating MPC parties, and (iii) obfuscated routing that reduces metadata leakage during bid submission and MPC setup \parencite{veilid_developer_book}. Because Veilid is blockchain-agnostic, settlement can occur off-overlay (Monero stagenet) while \emph{coordination} and \emph{communication} remain private and decentralized—aligning with the goals of sealed-bid privacy, front-running resistance, and auctioneerless operation \parencite{cryptonote2013_whitepaper}.

\subsection*{Project Goal}
Build a \textbf{peer-to-peer sealed-bid marketplace} that (1) preserves bidder privacy by default, (2) mitigates front-running risk during bid submission, and (3) operates \emph{without a central auctioneer}. The prototype combines MPC for winner/price computation with \emph{Veilid} for identity, routing, and availability \parencite{veilid_developer_book}. Payments clear on \emph{Monero stagenet} after the MPC outcome is published—exercising the full flow without handling real funds—while inheriting the confidential-transaction model and unlinkability properties from the CryptoNote design \parencite{cryptonote2013_whitepaper}.

\subsection{Project Objectives}
\begin{itemize}[noitemsep]
  \item \textbf{Main objective.} This project shall deliver a working peer-to-peer marketplace application on the Veilid network that exclusively supports sealed-bid listings and private content unlock for the winning bidder.

  \item \textbf{Testable objectives.}
    \begin{itemize}[noitemsep]
      \item \emph{Listings and purchases.} Implement end-to-end flows: create listing → submit bid → determine winner → complete purchase. \textbf{Test:} demo run and automated integration tests covering success/edge cases (invalid bid, tie, timeout).
      \item \emph{Sealed-bid via MPC.} Integrate a multi-party computation protocol to select the highest valid bid without revealing non-winning bids. \textbf{Test:} unit tests with mocked parties; property tests showing loser-bid privacy; reproducible benchmark for N bidders.
      \item \emph{Encrypted content unlock.} Ensure listing content remains encrypted; only the winner obtains the decryption key/hash after settlement. \textbf{Test:} verify ciphertext remains inaccessible to non-winners; successful decrypt by winner; tamper tests.
      \item \emph{Settlement on Monero Stagenet.} Simulate network init, funded wallets, and programmatic “real” transactions for deposits/escrow/release. \textbf{Test:} scripted stagenet transactions with confirmations; failure/reorg handling.
    \end{itemize}
\end{itemize}



\subsection{Project Beneficiaries}
Identify who benefits from this project (users, researchers, organizations, etc.) and how they benefit.

\subsection{Project Plan}
Provide a high-level timeline of the project. You can use a list or table if preferred.

\begin{itemize}[noitemsep]
    \item Placeholder0
    \item Placeholder1
    \item Placeholder2
    \item Placeholder3
\end{itemize}

\subsection{Risks Affecting the Project}
Outline any technical, logistical, or ethical risks that may affect project success, and how you plan to mitigate them.

\subsection{Legal, Social, Ethical and Professional Considerations}
Discuss any relevant ethical or legal implications of your project (e.g., data privacy, bias, intellectual property, accessibility).

\subsection{References}
\printbibliography

\newpage
\section{Research Ethics Checklist}
Summarize any ethical considerations and indicate whether formal ethics approval is needed. If applicable, reference consent forms or procedures.

\newpage
\section{Client Information Sheet (External Client Projects Only)}
Provide details about the client organization, contact person, and nature of collaboration.

\newpage
\section{Appendix: Use of Generative AI (if applicable)}
If AI tools (e.g., ChatGPT, Copilot) were used in preparing this document or project materials, describe exactly how and to what extent they were used.

\newpage
\section{Appendix: Legal, Social, Ethical, and Professional Issues (LSEPI)}
For Category 3 projects, provide a more detailed analysis of relevant LSEPI topics.

\end{document}
