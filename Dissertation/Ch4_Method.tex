% Chapter 4: Method — Handbook 8.2.5
% Detailed report of analysis, design, implementation, evaluation.
% ~2500 words for BSc. Be specific to YOUR project, not generic SE theory.
% State which lifecycle you followed, don't explain what it is.
% Must be clear about reused materials (see also Appendix B).

\section{Method}
\label{sec:method}

% TODO (~2500 words)

\subsection{Development Approach}
\label{sec:method-approach}

% - Iterative, prototype-driven development (state, don't explain the methodology)
% - Devnet-first: local Docker environment before any public network
% - Mock-driven testing: trait-based dependency injection for deterministic simulation
% - Tools: Rust 2021, cargo + nextest, Git, Docker Compose

\subsection{System Architecture}
\label{sec:method-architecture}

% - Two-tier coordination design:
%   - AuctionLogic<D,T,M,C>: generic over DhtStore, MessageTransport, MpcRunner, TimeProvider
%   - AuctionCoordinator: concrete Veilid implementation wrapping AuctionLogic
% - Justification: testability — full N-party auction simulations run in <1s with mocks
% - Module layout: traits/ -> mocks/ -> veilid/ -> marketplace/ -> mpc/ -> app/

% Include: high-level architecture diagram (Figure)

\subsection{Technology Choices and Justifications}
\label{sec:method-tech}

% Each choice needs a WHY and what alternatives were considered.

\subsubsection{Veilid for Coordination and Transport}
% - DHT-native (no separate DHT needed), private routes, no exit nodes
% - Directly satisfies project brief
% - Tradeoff: less battle-tested than Tor, smaller anonymity set

\subsubsection{MASCOT Protocol for MPC}
% - N-party support with dishonest-majority security (no honest-majority assumption)
% - MP-SPDZ provides production-grade implementation with mascot-party.x
% - Advantage over Shamir: tolerates N-1 corrupt parties; supports N >= 2
% - Tradeoff: higher computational cost than Shamir (OT-based), acceptable for auction workloads

\subsubsection{Commitment Scheme for Bid Integrity}
% - SHA256(bid_value || 32-byte nonce): hiding + binding
% - Simple, well-understood, no additional crypto infrastructure needed
% - Alternative considered: Pedersen commitments (homomorphic but more complex)

\subsubsection{Serialization Formats}
% - CBOR (ciborium) for DHT-stored records: self-describing, forward-compatible
% - Bincode for wire messages: compact, fast for ephemeral auction messages
% - Veilid DHT takes raw Vec<u8> — format choice is application-level

\subsubsection{Desktop UI Framework}
% - Dioxus 0.7.2: Rust-native, cross-platform desktop
% - Allows single-language codebase (Rust end-to-end)

\subsection{MPC Protocol Design}
\label{sec:method-mpc}

% - auction_n.mpc: N-party sealed-bid auction
% - Party assignment: by bid timestamp ascending (seller = party 0, always earliest)
% - Privacy model: party 0 sees winner ID + bid; others see only won: 0/1
% - Post-MPC verification: seller-initiated challenge-response
% - Decryption key transfer via Veilid private routes

\subsection{Testing Strategy}
\label{sec:method-testing}

% - Unit tests: traits, serialization, time/random providers
% - Integration tests: MultiPartyHarness with shared mock DHT/transport/MPC
%   - 3-party, 5-party, 10-party auctions
%   - Edge cases: ties, reserve price, expired listings, duplicate bids
% - E2E tests: real Veilid devnet + Docker, marked #[ignore]
% - No human participants (as per ethics form)

\subsection{Reused Software}
\label{sec:method-reuse}

% Required by Handbook 8.2.5 and 9.6. See also Appendix B.
% State what was reused and how it affected development.
%
% | Component       | Licence       | How used                          |
% |-----------------|---------------|-----------------------------------|
% | veilid-core     | MPL-2.0       | P2P networking, DHT, routing      |
% | MP-SPDZ         | BSD-3-Clause  | MPC protocol execution            |
% | Dioxus 0.7.2    | MIT/Apache    | Desktop UI framework              |
% | tokio           | MIT           | Async runtime                     |
% | sha2, aes-gcm   | MIT/Apache    | Cryptographic primitives          |
% | ciborium        | MIT/Apache    | CBOR serialization                |
